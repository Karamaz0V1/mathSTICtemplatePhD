\chapter{TITVS LVCRETIVS CARVS: DE RERVM NATVRA}

\minitoc %% for the short toc at the chapter beginning

If you get bored while writing your manuscript, you may read Lucretius.
Source: \href{http://www.thelatinlibrary.com/lucretius.html}{thelatinlibrary}.


\section{Liber I (first sentence)}

Aeneadum genetrix, hominum divomque voluptas,
alma Venus, caeli subter labentia signa
quae mare navigerum, quae terras frugiferentis
concelebras, per te quoniam genus omne animantum
concipitur visitque exortum lumina solis:
te, dea, te fugiunt venti, te nubila caeli
adventumque tuum, tibi suavis daedala tellus
summittit flores, tibi rident aequora ponti
placatumque nitet diffuso lumine caelum.


\section{Liber II (first sentence)}

Suave, mari magno turbantibus aequora ventis
e terra magnum alterius spectare laborem;
non quia vexari quemquamst iucunda voluptas,
sed quibus ipse malis careas quia cernere suavest.
suave etiam belli certamina magna tueri              
per campos instructa tua sine parte pericli;               
sed nihil dulcius est, bene quam munita tenere               
edita doctrina sapientum templa serena,
despicere unde queas alios passimque videre
errare atque viam palantis quaerere vitae,               
certare ingenio, contendere nobilitate,
noctes atque dies niti praestante labore
ad summas emergere opes rerumque potiri.


\section{Liber III (first sentence)}

E tenebris tantis tam clarum extollere lumen
qui primus potuisti inlustrans commoda vitae,
te sequor, o Graiae gentis decus, inque tuis nunc
ficta pedum pono pressis vestigia signis,
non ita certandi cupidus quam propter amorem
quod te imitari aveo; quid enim contendat hirundo
cycnis, aut quid nam tremulis facere artubus haedi
consimile in cursu possint et fortis equi vis?


\section{Liber IV (first sentence)}

Avia Pieridum peragro loca nullius ante
trita solo. iuvat integros accedere fontis
atque haurire, iuvatque novos decerpere flores
insignemque meo capiti petere inde coronam,
unde prius nulli velarint tempora musae;
primum quod magnis doceo de rebus et artis
religionum animum nodis exsolvere pergo,
deinde quod obscura de re tam lucida pango
carmina musaeo contingens cuncta lepore.

\section{Liber V (first sentence)}

Quis potis est dignum pollenti pectore carmen
condere pro rerum maiestate hisque repertis?


\section{Liber VI (first sentence)}

Primae frugiparos fetus mortalibus aegris
dididerunt quondam praeclaro nomine Athenae
et recreaverunt vitam legesque rogarunt
et primae dederunt solacia dulcia vitae,
cum genuere virum tali cum corde repertum,
omnia veridico qui quondam ex ore profudit;
cuius et extincti propter divina reperta
divolgata vetus iam ad caelum gloria fertur.

\newpage
\section{A last section}

Just so that you see the default header added by \LaTeX:
section name on front side and chapter name on back side.
